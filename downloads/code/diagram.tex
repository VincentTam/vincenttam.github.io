%        File: diagram.tex
%     Created: 六 3月 29 09:00 下午 2014 H
% Last Change: 六 3月 29 09:00 下午 2014 H
%
\documentclass[a4paper,12pt]{article}
\usepackage{amsmath}
\usepackage{tikz}
\usepackage{tikz-qtree}
\title{A \LaTeX{} Sample Diagram}
\author{Vincent Tam}
\date{March 29, 2014}
\begin{document}
\maketitle

I divide my answer into sections so as to see the layout of this
typesetting engine.

\section{Introductory rubbish}

There's no denying that combinatorics is a fine art of counting.
Therefore, questions related to this branch of math is hard for
students who don't like math.

The objective of this article is to test my skills in drawing simple
diagrams in \LaTeX{}, as well as to clarify the conceptual difference
between $_{10}C_{2}$ and $_{10}C_{1} \times {_{9}C_{1}}$. 

\section{So-called ``analysis'' to the problem}

The meaning of $_{10}C_{2}$ is easy to imagine.  What's left is to
think of a way to perceive $_{10}C_{1} \times _{9}C_{1}$.  A visual way
to understand the problem is good.

However, ``10'' is \emph{too big} for drawing such diagram.  Thus, I
reduce ``10'' to ``4''.

\section{The diagram}

\begin{center}
  \tikzset{grow'=right}
  \tikzset{every tree node/.style={anchor=base west}}
  \Tree [ .O
  [ .A [ .B ] [ .C ] [ .D ] ]
  [ .B [ .A ] [ .C ] [ .D ] ]
  [ .C [ .A ] [ .B ] [ .D ] ]
  [ .D [ .A ] [ .B ] [ .C ] ]
  ]
\end{center}

\section{Actual answer}

To see the difference better, I use an unordered list so that the
following 2 sentences won't be put into the same paragraph.

\begin{itemize}
  \item From the above diagram, the \emph{order} of picking 2 things
    \emph{does count}.
  \item From the equation below, the \emph{order} of picking 2 things
    \emph{doesn't count}.
    In $_{10}C_{2}$, we have
    \[
    \binom{10}{2} = \frac{9 \cdot 8}{2!}.
    \]
\end{itemize}

\end{document}


% vim:sw=2:tw=70:wrap
