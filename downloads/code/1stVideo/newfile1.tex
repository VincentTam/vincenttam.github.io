%        File: newfile1.tex
%     Created: 日 8月 11 10:00 下午 2013 H
% Last Change: 日 8月 11 10:00 下午 2013 H
%
\documentclass[a4paper]{article}
\usepackage[]{amsmath}
\usepackage[american]{babel}
\usepackage{csquotes}
\usepackage[backend=biber,style=apa,apabackref=true]{biblatex}
\DeclareLanguageMapping{american}{american-apa}
\addbibresource{mybib.bib}
\usepackage{url}
\urlstyle{tt}
\usepackage[pdftex,hidelinks]{hyperref}
\title{Efficient \LaTeX{} Editing with Vim}
\author{David Wong}
\date{\today}
\begin{document}
\maketitle
Vim has lots of advantages over other editors and is useful for C/C++ programming. \parencite{lam}
\section{Writing equations quickly and accurately with the \textsf{amsmath} package}
Some keyboard shortcuts in Latex-Suite: \parencite{goerz2008}
\begin{itemize}
  \item \verb|F5| gives you 8 options.
  \item Three-letter acronym:\\
    e.g. Typing \verb|EIT| gives you
    \begin{verbatim}
    \begin{itemize}
      \item 
    \end{itemize}<++>
    \end{verbatim}
  \item Math
    \begin{itemize}
      \item Typing \verb|`/| gives you \verb|\frac{}{<++>}<++>|
      \item Typing \verb|`2| gives you \verb|\sqrt{}<++>|
    \end{itemize}
  \item Quick jumping to \verb|<++>| by typing \verb|Ctrl+j|
\end{itemize}
Let's see some real examples.
\begin{equation}
  e^{i\pi}+1=0
  \label{eq:euler}
\end{equation}
Equation \eqref{eq:euler} was discovered by Euler.
\begin{align}
  1+\int_{0}^{\frac{\pi}{2}} \sin x dx &= \sum_{i=0}^{\infty} \left(\frac{1}{2}\right)^i \label{eq:eq2}\\
  &= 1 + \frac{1}{2} + \left(\frac{1}{2}\right)^2 + \left(\frac{1}{2}\right)^3 + \cdots \label{eq:eq3}
\end{align}
A university student wrote equation \eqref{eq:eq2}. High school students find \eqref{eq:eq3} easier.
\section{Easy citing in APA style with the \textsf{biblatex} package}
\textcite{oualline2001} thinks that Vim is the best editor. Juan A. Navarro (\citeyear{citeurl2013}) can cite online information quickly. Doing so on Google scholar is even easier. Doing exercises is good for your health. \parencite{wilmore2004physiology}
\printbibliography
\end{document}


