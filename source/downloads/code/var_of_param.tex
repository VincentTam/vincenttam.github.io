%        File: var_of_param.tex
%     Created: Sat Oct 26 11:00 AM 2013 H
% Last Change: Sat Oct 26 11:00 AM 2013 H
%
\documentclass[a4paper,12pt]{article}
\usepackage[a4paper]{geometry}
\usepackage[sumlimits]{amsmath}
\usepackage{mathtools}
\usepackage[]{url}
\usepackage[]{hyperref}
\renewcommand{\theequation}{\roman{equation}}
\newcommand{\ud}{\,\mathrm{d}}
\author{Vincent Tam}
\title{Explanation to Wikipedia's Variation of Parameter}
\begin{document}
\maketitle
Wikipedia's explanation is too concise. \cite{wiki13}

First, you have
\begin{equation}
  y^{(n)}(x)+\sum_{k=0}^{n-1} a_k(x)y^{(k)}(x)=b(x).
  \label{eq:non-homo}
\end{equation}
Here, $y$ means the general solution of \eqref{eq:non-homo}.

I just use the notations in Wikipedia: For $i$ between 1 and $n$,
inclusive, $y_i$ is the solution of the homogenous equation
\begin{equation}
  y_i^{(n)}(x)+\sum_{k=0}^{n-1} a_k(x)y_i^{(k)}(x)=0.
  \label{eq:homo}
\end{equation}
Unlike Wikipedia, in equation \eqref{eq:homo}, the $i$ under the
summation sign is changed to $k$ so as to avoid confusion of symbols.
In other words, \textbf{we use $i$ to represent $c_i(x)$ and $y_i(x)$,
while using $k$ to represent the coefficient of the terms in
\eqref{eq:non-homo} (i.e. $a_k(x)$) and the derivatives of $y_i(x)$ of
different orders. (i.e. $y_i^{(k)}(x)$}\\
Let $y_p$ be the particular solution for the non-homogeneous ODE.

We make an educated guess like this:
\begin{equation}
  \boxed{y_p(x)=\sum_{i=1}^{n} c_i(x) y_i(x)}
  \label{eq:part-def}
\end{equation}
where $c_i(x)$ is a differentiable function for each $i$.\\
By doing so, we have created $n$ unknowns (that is, $c_i(x),i=1,\dots
n$), and they are also known as \emph{freedom}. We may make use of the
freedom like this:
\begin{equation}
  y_p'(x)=\left[ \sum_{i=1}^{n} c_i'(x) y_i(x) \right]+\left[
  \sum_{i=1}^{n} c_i(x) y_i'(x) \right]
  \label{eq:part-d1}
\end{equation}
Continuing the differentiation gives complicated expression, so we use
our freedom to give some restrictions on $c_i(x),i=1,\dots n$. i.e. We
set
\begin{equation}
  \boxed{\sum_{i=1}^{n} c_i'(x) y_i(x)=0}
  \label{eq:part-d1-restr}
\end{equation}

Why is there such a condition? It is because we want to simplify
\eqref{eq:part-d1} by ruling out \emph{something}. (i.e. setting
something to \emph{zero}) What is \emph{something}?
\begin{enumerate}
  \item $y_i(x)=0 \text{ or } y_i'(x)=0 \text{ or } c_i(x)=0$ for each
    $i=1,\dots n$\\
    They are clearly unreasonable restrictions.
  \item $c_i'(x)=0$ for each $i=1,\dots n$\\
    Then $c_i(x)$ is a constant function for each $i=1,\dots n$. In
    other words, $y_p(x)=\sum_{i=1}^{n} c_i(x) y_i(x)$ is just linear
    combinations of $y_i(x)$. Remembering that $y_i(x)$ are denoted as
    the solution of \eqref{eq:homo}, we conclude that this restriction
    can't be accepted.
\end{enumerate}

Being too ``strong'', the above guesses fails. Thus, what we need to
rule out is \emph{either} one of the middle bracket in
\eqref{eq:part-d1}. Why \emph{must} it be the first one?

Otherwise assume that 
\begin{equation}
  \sum_{i=1}^{n} c_i(x) y_i'(x)=0,
  \label{eq:part-d1-restr0-e1}
\end{equation}
which leads us to
\begin{equation}
  y_p'(x)=\sum_{i=1}^{n} c_i'(x) y_i(x)
  \label{eq:part-d1-restr0-d1a}
\end{equation}

Continue the above process of differentiating $y_p(x)$.
\begin{equation}
  y_p''(x)=\left[ \sum_{i=1}^{n} c_i''(x) y_i(x) \right]+\left[
  \sum_{i=1}^{n} c_i'(x) y_i'(x) \right]
  \label{eq:part-d1-restr0-d2}
\end{equation}
Making assumptions similar to \eqref{eq:part-d1-restr0-e1}, we have
\begin{equation}
  \sum_{i=1}^{n} c_i'(x) y_i'(x)=0\text{; and}
  \label{eq:part-d1-restr0-e2}
\end{equation}
\begin{equation}
  y_p''(x)=\sum_{i=1}^{n} c_i''(x) y_i(x)
  \label{eq:part-d1-restr0-d2a}
\end{equation}

From \eqref{eq:part-d1-restr0-e1} and \eqref{eq:part-d1-restr0-e2}, we
know that if we carry out the process repeatedly, we get a system of
``linear'' equations with ``unknowns '' $y_i'(x)$ (which is actually
known since $y_i(x)$ have been found out from \eqref{eq:homo}.) The
most insensible thing of this ``system'' is the ``coefficient matrix''
with entries $c_i^{(k)}$ on the $k$-th row and the $i$-th column. For
the sake of clarity, the ``system'' is illustrated below.
\begin{equation*}
  \left\{\begin{aligned}
    c_1(x) y_1'(x) &+ c_2(x) y_2'(x) &+ \dots &+ c_n(x) y_n'(x) &= 0\\
    c_1'(x) y_1'(x) &+ c_2'(x) y_2'(x) &+ \dots &+ c_n'(x) y_n'(x) &=
    0\\ \vdots\\ c_1^{(k)}(x) y_1'(x) &+ c_2^{(k)}(x) y_2'(x) &+ \dots
    &+ c_n^{(k)}(x) y_n'(x) &= 0
  \end{aligned}\right.
  \tag{*}
  \label{eq:part-d1-restr0-sys}
\end{equation*}

We don't care about the range of $k$ such that the R.H.S. of the
last equation in the above system is zero, as well as the missing
equation which can be yielded after substituting $y_p^{(k)}(x)$ (for
example, equations \eqref{eq:part-d1-restr0-d1a} and
\eqref{eq:part-d1-restr0-d2a}) into \eqref{eq:non-homo}

Therefore, equations \eqref{eq:part-d1-restr0-e1} to
\eqref{eq:part-d1-restr0-sys} are all \emph{wrong}. Instead, we should
have
\begin{equation}
  \boxed{y_p'(x)=\sum_{i=1}^{n} c_i(x) y_i'(x)}
  \label{eq:part-d1a}
\end{equation}
\begin{equation}
  y_p''(x)=\left[ \sum_{i=1}^{n} c_i'(x) y_i'(x) \right]+\left[
  \sum_{i=1}^{n} c_i(x) y_i''(x) \right]
  \label{eq:part-d2}
\end{equation}
Again, we set
\begin{equation}
  \boxed{\sum_{i=1}^{n} c_i'(x) y_i'(x)=0}
  \label{eq:part-d2-restr}
\end{equation}
so that
\begin{equation}
  \boxed{y_p''(x)=\sum_{i=1}^{n} c_i(x) y_i''(x).}
  \label{eq:part-d2a}
\end{equation}

For each $1 \le k \le n-1$, we have
\begin{equation}
  \boxed{\sum_{i=1}^{n} c_i'(x) y_i^{(k-1)}(x)=0}
  \label{eq:part-dk-restr}
\end{equation}
and
\begin{equation}
  \boxed{y_p^{(k)}(x)=\sum_{i=1}^{n} c_i(x) y_i^{(k)}(x).}
  \label{eq:part-dka}
\end{equation}

Why \emph{can't} we apply \eqref{eq:part-dk-restr} and
\eqref{eq:part-dka} to the case of $k=n$?\\
Differentiating \eqref{eq:part-dka} with respect to $x$, we get
\begin{equation}
  y_p^{(n)}(x)=\left[ \sum_{i=1}^{n} c_i'(x) y_i^{(n-1)}(x)
  \right]+\left[ \sum_{i=1}^{n} c_i(x) y_i^{(n)}(x) \right]
  \label{eq:part-dn}
\end{equation}
Recall from \eqref{eq:non-homo} that we have
\begin{equation}
  y_p^{(n)}(x)+\sum_{k=0}^{n-1} a_k(x) y_p^{(k)}(x)=b(x).
  \label{eq:sub1}
\end{equation}
Substitute \eqref{eq:part-dka} and \eqref{eq:part-dn} into
\eqref{eq:sub1},
\begin{flalign*}
  &\left[ \sum_{i=1}^{n} c_i'(x) y_i^{(n-1)}(x) \right]+\left[
  \sum_{i=1}^{n} c_i(x) y_i^{(n)}(x) \right]+\\
  &\quad \left\{ \sum_{k=0}^{n-1} a_k(x) \left[ \sum_{i=1}^{n} c_i(x)
  y_i^{(k)}(x) \right] \right\}=b(x).&\\
  &\left[ \sum_{i=1}^{n} c_i'(x) y_i^{(n-1)}(x) \right]+\left[
  \sum_{i=1}^{n} c_i(x) y_i^{(n)}(x) \right]+\\
  &\quad \left\{ \sum_{i=1}^{n} c_i(x) \left[ \sum_{k=0}^{n-1} a_k(x)
  y_i^{(k)}(x) \right] \right\}=b(x).\nonumber&\\
  &\left[ \sum_{i=1}^{n} c_i'(x) y_i^{(n-1)}(x) \right]+\left\{
  \sum_{i=1}^{n} c_i(x) \left[ y_i^{(n)}+\sum_{k=0}^{n-1} a_k(x)
  y_i^{(k)}(x) \right] \right\}=b(x)\\
  &\left[ \sum_{i=1}^{n} c_i'(x) y_i^{(n-1)}(x) \right]+\left[
  \sum_{i=1}^{n} c_i(x) (0) \right]=b(x)
\end{flalign*}
Thus, we have
\begin{equation}
  \boxed{\sum_{i=1}^{n} c_i'(x) y_i^{(n-1)}(x)=b(x)}
  \label{eq:part-dn-restr}
\end{equation}
From equations \eqref{eq:part-dk-restr} and \eqref{eq:part-dn-restr},
\begin{equation*}
  \left\{\begin{aligned}
    c_1'(x) y_1(x) &+ c_2'(x) y_2(x) &+ \dots &+ c_n'(x) y_n(x) &= 0\\
    c_1'(x) y_1'(x) &+ c_2'(x) y_2'(x) &+ \dots &+ c_n'(x) y_n'(x) &=
    0\\
    \vdots\\
    c_1'(x) y_1^{(n-2)}(x) &+ c_2'(x) y_2^{(n-2)}(x) &+ \dots &+
    c_n'(x) y_n^{(n-2)}(x) &= 0\\
    c_1'(x) y_1^{(n-1)}(x) &+ c_2'(x) y_2^{(n-1)}(x) &+ \dots &+
    c_n'(x) y_n^{(n-1)}(x) &= b(x)
  \end{aligned}\right.
  \tag{$\heartsuit$}
  \label{eq:part-sys}
\end{equation*}

By Cramer's Rule,
\[
  c_i'(x)=\frac{W_i(x)}{W(x)},
\]
where $W(x)$ is the Wronskian of $y_i(x)$ and $W_i(x)$ is the
determinant of replacing the $i$-th column of $W(x)$ by
$(0,0,\dots,0,b(x))$. Hence,
\begin{equation}
  c_i(x)=\int \frac{W_i(x)}{W(x)} \ud x.
  \label{eq:ans}
\end{equation}
Substitute \eqref{eq:ans} into \eqref{eq:part-def},
\[
  \boxed{y_p(x)=\sum_{i=1}^{n} y_i(x) \int \frac{W_i(x)}{W(x)} \ud x}
\]
\begin{thebibliography}{9}
  \bibitem[wiki12]{wiki13}
    Variation of Parameters. (n.d.). In \emph{Wikipedia}. Retrieved
    October 26, 2013, from
    \url{https://en.wikipedia.org/wiki/Variation_of_parameters}
\end{thebibliography}
\end{document}

% vim:tw=70:wrap
